\documentclass{amsart}
\usepackage{hyperref}
\usepackage[draft]{fixme}
\usepackage{xcolor}
\usepackage{mathrsfs}

\newenvironment{grayed}{\color{gray}}{\ignorespacesafterend}

\newtheorem{thm}{Theorem}
\newtheorem{lem}{Lemma}
\newtheorem{prop}{Proposition}
\newtheorem{defn}{Definition}
\newtheorem{rem}{Remark}

\newcommand{\setcond}[2]{\left\{ #1 \mid #2 \right\}}
\DeclareMathOperator{\xlim}{xlim}
\DeclareMathOperator{\SUPER}{SUPER}
\newcommand{\eqdef}{\overset{\mathrm{def}}{=}}
\newcommand{\subsets}{\mathscr{P}}
\newcommand{\norm}[1]{\left\lVert #1\right\rVert}
\newcommand{\abs}[1]{\left\lvert #1\right\rvert}
\newcommand{\supfun}[1]{\left\langle #1\right\rangle}

\hypersetup{pdftitle={A Proof of Existence of Smooth Classical Solutions of Navier-Stokes Equations},
 pdfauthor={Victor Porton},
 pdfsubject={Navier-Stokes equations},
 pdfkeywords={Navier-Stokes equations existence and smoothness, Navier-Stokes Clay Math Millennium Prize Problem, PDE, partial differential equations}}

\begin{document}

\noindent
Ad: \href{https://science-dao.org}{Donate for science.}

The most recent version of this document is available \href{https://github.com/vporton/navier-stokes}{here}.

\title[Navier-Stokes]{A Proof of Existence of Smooth Classical Solutions of Navier-Stokes Equations}

\author{Victor Porton, ORCID 0000-0001-7064-7975}

\email{\href{mailto:mailto:porton.victor@gmail.com}{porton.victor@gmail.com}}

\urladdr{\href{https://math.portonvictor.org}{https://math.portonvictor.org}}

\date{\today}

\begin{abstract}
A proof of existence of classical smooth solutions of the Navier-Stokes Millennium Prize problem, by usage of my general topology theorem proved before. This constitutes a partial solution of the Millennium Prize problem (no proof of boundness of energy is provided).
\end{abstract}

\subjclass[2020]{35Q30, 35A01, 35A09, 35A25, 35D99}

\maketitle

\section{Introduction}

This is a rectified proof by ChatGPT~\cite{navier-stokes-chat} of the Navier-Stokes Clay Math Millennium Prize Problem~\cite{navier-stokes-clay}, using the approach of generalized limits~\cite{limit} (more general generalized functions than distributions). In fact, ChatGPT didn't prove the energy inequality, but only existence of solutions, despite it claimed~\cite{navier-stokes-chat} a full solution. I also don't provide a proof of energy inequality.

In my approach multiplication of every functions is defined, what allows to write formulas undefined for distributions.

So, I did a half of the work (the theorem that $\lim$ (limit) functional can be linearly extended to the space of all functions and that on the space of extended limits all continuous operations can be extended with preservation of all algebraic identities). And ChatGPT did a half of work (applied known methods with addition of my this theorem).

Extraordinary claims need extraordinary evidence. The claim is a partial Millennium Prize Problem solution. The evidence of this ``functional extension'' theorem is an extraordinary evidence, because it is a new very powerful theorem in general topology and its proof is extraordinarily longer than a noob would expect.

I had a little trouble to understand the ChatGPT's proof, because while being the world-best expert in general topology, I am no expert in differential equations. But I was able to write down my more detailed (and in parts more elegant than ChatGPT's) proof.

\section{Problem statement}

We are going to prove existence (in the sense of classical real-number solutions) and smoothness~($C^\infty$) of solutions to Navier-Stokes equations on $\mathbb{R}^N$ ($N\in\mathbb{N}$).
Let $u:\mathbb{R}^N\times[0;\infty)\to\mathbb{R}^N$ and pressure $p(x,t)$.
\begin{align*}
\partial_t u - \nu\Delta u + (u \cdot \nabla) u + \nabla p &= f ;\\
\nabla \cdot u &= 0;\\
u({\cdot},0) &= u_0
\end{align*}
for viscosity $\nu>0$, $f(t,x)\in C^\infty$ and initial condition $u_0\in C^\infty$.

\section{The limit extension theorem}

\begin{thm}
The $\lim$ (limit) functional can be linearly extended to the space of all functions and on the space of extended limits all continuous operations can be extended with preservation of all algebraic identities.
\end{thm}

\begin{proof}
This is a summary of several theorems from~\cite{limit}.
\end{proof}

\section{Continuous functions with values in $\SUPER(\mathbb{R})$}

Let $X$ be some Lebesgue measurable set and some element $0\in X$ (not necessarily real zero or zero vector).

\begin{defn}
Let $a,b\in\SUPER(\mathbb{R})$.
\[ a\le b \eqdef \exists f,g:X\to\mathbb{R}:\left(\lim_{x\to 0}f(x)=a\land\lim_{x\to 0}g(x)=b\land f\le g\right). \]
\end{defn}

\begin{prop}
The above defines is a partial order on~$\SUPER(\mathbb{R})$.
\end{prop}

\begin{proof}
Transitivity and reflexivity are obvious.

For antisymmetry, let $a\leq b$ and $b\leq a$. We have $\lim_{x\to 0}|g(x)-f(x)|\geq 0$ and $\lim_{x\to 0}|g(x)-f(x)|\leq 0$. So, $\lim_{x\to 0}|g(x)-f(x)|=0$ and therefore $a=b$.
\end{proof}

\begin{thm}
For each $S\in\subsets\SUPER(\mathbb{R})$, such that $\forall a\in S:a\ge B$ for some constant $B\in\mathbb{R}$, there exists $\inf S\in\SUPER(\mathbb{R})$.
\end{thm}

\begin{proof}
Choose
\[ F\in\prod_{a\in S}\setcond{f:X\to\SUPER(\mathbb{R})}{\lim_{x\to 0}f(x)=a}. \]
Let
\[ m=\lim_{x\to 0}\left(x\mapsto\inf_{f\in F}\max\{f(x),B\}\right). \]
($\inf_{f\in F}\max\{f(x),B\}$ exists because this set is bounded from below.) We will show that $m$ is the infimum of~$S$.

It's obvious that $\forall a\in S:m\le a$ and $\forall a\in S:x\le a\Rightarrow x\le m$. 
\end{proof}

The following theorem allows to extend Lebesgue integrals to the space of generalized limits functions in an easy way:

\begin{thm}
If a set $X$ is Lebesgue measurable and $f:X\to\SUPER(\mathbb{R})$ is continuous, then:
\begin{enumerate}
\item $\setcond{x\in X}{f(x)\ne 0}$ is partitioned into Lebesgue measurable sets (\emph{zones})
  \[ \setcond{Z_k}{k\in\SUPER(\mathbb{R})\setminus\{0\}}, \]
  where zone
  \[ Z_k=\setcond{x\in X}{\exists a\in\mathbb{R}\setminus\{0\}:f(x)=ka}. \]
\item Each $Z_k$ is an open set in~$X$.
\item $f(x)=0$ for each $x$ on the boundary $X\cap\partial Z$ of each zone.
\item\label{as-func-mult} $f(x)=k_Z g_Z(x)$ for each $x\in X\cap\overline{Z}$ for some $k_Z\in\SUPER(\mathbb{R})\setminus\{0\}$ and $g_Z:X\cap\overline{Z}\to\mathbb{R}$ for each zone $Z$.
\end{enumerate}
\end{thm}

\begin{proof}
To prove that $f(a)=0$ on boundaries of zones, consider two only possible cases for $a\in X\cap\partial Z_k$:
\begin{enumerate}
\item ``$a\in\overline{Z_k}$ and $\forall\epsilon>0\exists x\in\mathbb{R}:(\abs{a-x}<\epsilon\land f(x)=0)$.'' This is possible only if $f(a)=0$ due to continuity of~$f$.

\item ``$a\in\overline{Z_k}$ and $\forall\epsilon>0\exists x\in\mathbb{R}:(\abs{a-x}<\epsilon\land f(x)=k'g(x))$, where $k'\in\SUPER(\mathbb{R})\setminus\{0\}$ and $g:X\cap\overline{Z'}\to\mathbb{R}$ where zone $Z'\ne Z_k$.''
$f(x)$ belongs to the same zone $Z_k$, unless $f(a)=0$, because otherwise we would have $kf(x)=k'f(x)$ for $k$ and~$k'$ from different zones $Z_k$ and~$Z'$, respectively.
\end{enumerate}

$Z_k$ doesn't intersect $\partial Z_k$ and is, therefore, open. Obviously zones don't intersect and their union is $\setcond{x\in X}{f(x)\ne 0}$.

Item~\ref{as-func-mult} is obvious.

We have zones (without loss of generality) being open sets and therefore Lebesgue measurable.
\end{proof}

\section{Lebesgue integrals}

\begin{rem}
(Lebesgue) integrals could be instead defined on the space of generalized limits functions the same way as Lebesgue integrals on the space of functions with real number values. This requires existence of (possibly infinite) supremum of a set of supersingularities. But for our purposes it is enough to define integrals on the space of continuous generalized limits functions, and this special case is much simpler to prove properties of.
\end{rem}

Define Lebesgue integrals on the space of continuous generalized limits functions as \[ \int_A f(x) dx = \sum_{Z\in S}k_Z\int_Z a_Z(x) dx \] if each Lebesgue integral on the right side exists and the sum is absolutely convergent (and so does not depend on the order of summation), where $S$~is a set of zones.

It is easy to show that this definition does not depend on the choice of~$k_Z$ and~$a_Z$.

\begin{thm}
If an integral is absolutely convergent on a measurable set, then it is convergent.
\end{thm}

\begin{proof}
Obvious consequence of the same property for Lebesgue integrals on the space of functions with real number values.
\end{proof}

\subsection{Properties of Lebesgue integrals}

\begin{thm}
The fundamental theorem of calculus holds for integrals of continuous functions on the space of generalized limits functions:

Let $a<b$ be real numbers. Then:
\begin{enumerate}
\item If $f\in L^1([a,b])$ (absolutely integrable by Lebesgue) and
\[ F(x) = \int_a^x f(t) dt \quad\text{for } x\in [a,b]\]
then $F$ is absolutely continuous on $[a,b]$, differentiable almost everywhere on $[a,b]$, and $F'(x) = f(x)$ almost everywhere on $[a,b]$.
\item $G:[a,b]\to\SUPER(\mathbb{R})$ is absolutely continuous on $[a,b]$, $G'$ exists almost everywhere, and $G'(x) \in L^1([a,b])$ and
\[ G(x)-G(a) = \int_a^x G'(t) dt \quad\text{for } x\in [a,b].\]
\end{enumerate}
\end{thm}

\begin{proof}
Obvious consequence of the same property for Lebesgue integrals on the space of functions with real number values (we use an absolutely convergent series).
\end{proof}

\section{Inhomogenious linear differential equations}

Theorem elaborated from~\cite{duhamel-chat}.

\begin{defn}
\emph{Inhomogeneous linear evolution} with~$\Delta$ as the differential operator is
\begin{equation*}
\left\{
\begin{aligned}
(\partial_t - \Delta)u &= f;\\
u({\cdot},0) &= 0.
\end{aligned}
\right.
\end{equation*}
\end{defn}

\begin{thm}[First order Duhamel's principle]
Define auxiliary equations
\begin{equation*}
\left\{
\begin{aligned}
(\partial_t - \Delta)u_s &= 0;\\
u_s(x,s) &= f(x,s).
\end{aligned}
\right.
\end{equation*}
If the above family of equations have continuous in~$s$ solutions~$u_s(x,t)$ on supersingularities then the function
\[ u(x,t) = \int_0^t u_s(x,t) ds \]
is a continuous solution of the inhomogeneous linear evolution equation in supersingularities.
\end{thm}

\begin{proof}
Continuity of $u_s$ makes the integral defined.

$u({\cdot},0) = 0$ is obvious.

By the Leibniz rule for an integral with moving upper limit,
\[ \partial_t u(x,t) = \partial_t \int_0^t u_s(x,t) ds = u_t(x,t) + \int_0^t \partial_t u_s(x,t) ds. \]

$(\partial_t - \Delta)u(x,t) = \partial_t \int_0^t u_s(x,t) ds - \Delta \int_0^t u_s(x,t) ds =
u_t(x,t) + \int_0^t (\partial_t u_s(x,t) - \Delta u_s(x,t)) ds = f(x,t)$, so $(\partial_t - \Delta)u = f$.
\end{proof}

\section{Rewriting in integral form}

% The below proof assures uniqueness only of continuous solutions (because the integral requires a continuous argument). I didn't prove uniqueness of solutions. However, if at some point of time~$t_{\operatorname{crit}}$, the solutions ``split'' into both continuous and non-continuous~$u$, then it contradicts to a known theorem\fxnote{Cite.}, stating that under initial condition at~$t_{\operatorname{crit}}$ solutions are preserved to be smooth\fxwarning{Seems to be a wrong logic, because at~$t_{\operatorname{crit}}$ this solution can take only the initial value.}, so by contradiction we have full Clay Math problem solution.

% If at some point of time~$t_{\operatorname{crit}}$ speed~$u$ becomes non-continuous,

% Let's introduce a Banach norm on~$\mathbb{R}^N$:\fxwarning{Unused.}
% \[ \lVert y\rVert_2 = \sqrt{\sum_{i=1}^N y_i^2}. \]
% \[ \lVert y\rVert = \sqrt{\sup\setcond{\limsup_{x\to 0} f(x)^2}{f:\mathbb{R}^N\to\mathbb{R}, \lim_{x\to 0} f(x) = y}}. \]

Let denote Leray projection as~$\mathbb{P}$.

\begin{lem}
If $u$ is a $C^\infty$ super-sin\-gu\-lar function, then \[ e^{\nu (t-s) \Delta} \mathbb{P} \nabla \cdot (u \otimes u)(s) \] is also a continuous super-singular function for $s\in[0,t]$.
\end{lem}

\begin{proof}
Obvious.
\end{proof}

\begin{defn}
I call the following defined for continuous functions taking supersingularities as values:
\begin{equation}
u(t) = e^{\nu t \Delta} u_0 - \int_0^t e^{\nu (t-s) \Delta} \mathbb{P} \nabla \cdot (u \otimes u)(s) ds
+ \int_0^t e^{\nu (t-s) \Delta} \mathbb{P} f(s) ds
\end{equation}
\emph{supersingular mild solutions}\footnote{This formula is usually called \emph{mild (Duhamel) formulation}.} of Navier-Stokes equations.
\end{defn}

% \begin{grayed}
% Riesz transform of a classical function~$f$ is defined~\cite{enwiki:riesz} as
% \[
% \mathcal{R}_j f(x) = c \lim_{\epsilon\to 0}\int_{\mathbb{R}^N\setminus B_{\epsilon(0)}}\frac{(x_j-t_j)f(t)}{|x-t|^4} dt.
% \]
% where $c$ is a real constant.
% This obviously generalizes to supersingular functions.
% \end{grayed}

\begin{thm}
$C^\infty$ supersingular mild solutions of Navier-Stokes equations are exactly $C^\infty$ solutions of Navier-Stokes equations (without the requirement that~$p$ is $C^\infty$) taken on the space of supersingularities.
\end{thm}

\begin{proof}
It is known that \[ \widehat{\mathbb{P}f}(\xi) = \left(I - \frac{\xi \otimes \xi}{|\xi|^2}\right)\hat{f}(\xi). \]

$\mathbb{P}$ is obviously a linear operator.

$\mathbb{P}\nabla q=0$ for any scalar~$q$.

$\mathbb{P}$ commutes with $\Delta$ and with the heat semigroup: 
$\mathbb{P}\Delta=\Delta\mathbb{P}$,
$\mathbb{P}e^{\nu t \Delta}=e^{\nu t \Delta}\mathbb{P}$. (It's clear from these equalities in the classical case and linear combinations for the zones.)

Apply $\mathbb{P}$ to the PDE. Since $\mathbb{P}\nabla p=0$, we get
\[ \mathbb{P}\partial_t u - \nu\mathbb{P}\Delta u - \mathbb{P}((u\cdot\nabla) u) = \mathbb{P}f. \]

Because $u$ is divergence-free and $\mathbb{P}$ equals identity on divergence-free fields, $\mathbb{P}u=u$. $\mathbb{P}$ commutes with $\partial_t$ and $\Delta$.

Thus we have the consequence
\[ \partial_t u - \nu\Delta u + \mathbb{P}((u\cdot\nabla) u) = \mathbb{P}f. \]

We can write the nonlinear term as a divergence of a tensor:
\[ (u\cdot\nabla) u = \nabla \cdot (u\otimes u), \]
because $(u\otimes u)_{ij}=u_i u_j$ and $\nabla \cdot (u\otimes u) = \sum_j\partial_j (u_i u_j)$.

Thus the PDE becomes: \[ \partial_t u - \nu\Delta u + \mathbb{P}\nabla \cdot (u\otimes u) = \mathbb{P}f. \]

Rewrite this PDE as a system:
\[ \partial_t u - \nu\Delta u = -F(t) + \mathbb{P}f(t),\quad F(t) = \mathbb{P}\nabla \cdot (u\otimes u). \]

Denote $G_t = e^{\nu t\Delta}$.

For the linear homogeneous heat equation $\partial_t v - \nu\Delta v = 0$ with initial data $v(0)=v_0$ the solution is $v(t) = G_t v_0$.

For the inhomogeneous equation, the Duhamel principle (variation-of-constants formula) gives the unique solution as
\[ u(t) = G_t u_0 + \int_0^t G_{t-s} \bigl( -F(s) + \mathbb{P}f(s) \bigr) \, ds = G_t u_0 - \int_0^t G_{t-s} F(s) \, ds + \int_0^t G_{t-s} \mathbb{P}f(s) \, ds. \]

Substituting the value of~$F$ gives exactly the mild formula.

Now prove in the converse: if continuous~$u$ satisfies the mild equation, then it satisfies the PDE and we can recover a pressure~$p$.

Assume $u$ to be a continuous supersingular function.

Differentiate the mild equation in~$t$. Use facts:
\begin{itemize}
\item $\partial_t(e^{\nu t \Delta}u_0) = \nu \Delta e^{\nu t \Delta} u_0$;
\item $\frac{d}{dt}\int_0^t e^{\nu (t-s) \Delta} F(s) ds = F(t) + \int_0^t \nu e^{\nu (t-s) \Delta} F(s) ds$.
\end{itemize}
Apply this with $F(s) = \mathbb{P}\nabla \cdot (u\otimes u)(s)$. Differentiating mild gives:
\begin{multline*}
\partial_t u(t) = \nu \Delta e^{\nu t\Delta}u_0 - \mathbb{P}\nabla \cdot (u\otimes u)(t) + {} \\ \int_0^t \nu\Delta e^{\nu (t-s)\Delta} \mathbb{P}\nabla \cdot (u\otimes u)(s) ds + \mathbb{P}f(t) - \int_0^t \nu\Delta e^{\nu (t-s)\Delta} \mathbb{P}f(s) ds.
\end{multline*}
Group terms:
\[ \partial_t u(t) - \nu \Delta u(t) = \mathbb{P}\nabla \cdot (u\otimes u)(t) + \mathbb{P}f(t). \]
This is exactly the projected PDE.

Since $\mathbb{P}$ annihilates gradients, applying $(I-\mathbb{P})$ to the PDE recovers the pressure gradient term: start from the original PDE form:
\[ \partial_t u - \nu\Delta u + (u \cdot \nabla) u + \nabla p = f \]

We already have projection $\mathbb{P}$ of this equation satisfied. To obtain $\nabla p$ explicitly, take divergence of the PDE:
\[ \nabla \cdot \bigl(((u\cdot\nabla)u) + \nabla p - f\bigr) = 0 \]
because
\[ \nabla \cdot (\partial_t u - \nu\Delta u) = \partial_t(\nabla\cdot u) -\nu\Delta(\nabla\cdot u) = 0 \]
when $\nabla\cdot u=0$ initially and evolution preserves it (or check separately).

Use $\nabla\cdot(u\cdot\nabla)u = \sum_{i,j}\partial_i\partial_j(u_i u_j)$. Thus
\[ -\Delta p = \nabla \cdot \nabla (u\otimes u) - \nabla \cdot f\quad\text{(Poisson)}. \]
% i.e.
% \[ -\Delta p = \sum_{i,j}\partial_i\partial_j(u_i u_j). \]
It is common knowledge that the Poisson equation's solution is $C^\infty$ if $u$ and $f$ are $C^\infty$.

Because $u$ is smooth, we can split this equality into zones, and it is true in every zone.
On borders of zones, it can be constructed as a limit of the Poisson equation's solution in a zone. It does not depend on choice of the zone we take limit in, because~$u\in C^\infty$. Outside of zones it's zero.
So, we have supersingular solution, too.

% \begin{grayed}
% \fxwarning{Need a detailed solution in su\-per-sin\-gu\-lar case.}
% Solve the Poisson equation for $p(\cdot,t)$ (up to additive function of $t$) by convolution with the Newton potential: for each fixed $t$,
% \[
% p(\cdot,t) = (-\Delta)^{-1}\nabla\cdot\nabla\cdot(u\otimes u)(\cdot,t) = \mathcal{R}_i\mathcal{R}_j(u_i u_j),
% \]
% where $\mathcal{R}_k$ are the Riesz transforms. This recovers~$p$ (up to function of time) and shows that
% \[ \nabla p = (I-\mathbb{P})\nabla\cdot(u\otimes u), \]
% so the full PDE
% \[ \partial_t u - \nu\Delta u + \nabla p + \nabla\cdot(u\otimes u) = 0 \]
% holds. Since $\mathbb{P}\nabla p=0$, the splitting\fxnote{Check!} $\mathbb{P}\nabla\cdot(u\otimes u)-(I-\mathbb{P})\nabla\cdot(u\otimes u)=\nabla\cdot(u\otimes u)$ recovers the original nonlinear term plus the pressure.
% \end{grayed}
\end{proof}

\section{The sequence of approximate solutions}

Define the Picard iteration mapping~$\mathcal{T}$ on ti\-me-de\-pen\-dent $C^\infty$ vector fields by
\begin{equation*}
(\mathcal{T}v)(t) = G_t u_0 - \int_0^t G_{t-s}\mathbb{P}\nabla\cdot(v\otimes v)(s)ds + \int_0^t G_{t-s}\mathbb{P}f(s)ds.
\end{equation*}
So, we have a sequence
\[ u^{(0)}(t) = u_0,\quad u^{(n+1)}(t) = (\mathcal{T}u^{(n)})(t). \]

Note that the integrand is~$L^1$ because $\nabla\cdot(v\otimes v)$ is~$C^\infty$.

Each element of this sequence is a continuous supersingular function.

Try as the solution $u(t)=\lim_{n\to\infty}u^{(n)}(t)$ as the limit of this sequence. The limit exists as I show below.

\section{Interchange of limit and integration}

Let $\lim$ be our extended limit. On each (closed) zone (a closed interval), $\lim$ is a bounded linear operator. Therefore (because it is a linear bounded operator~\cite{int-banach} $\lim$~can be interchanged with integrals.)
It also maps absolutely convergent series to absolutely convergent series, therefore this extends to the entire closed interval. Thus we have:

\begin{multline*}
\lim_{n\to\infty}u^{(n)}(t) = G_t u(0) - \int_0^t \lim_{n\to\infty}\bigl(G_{t-s} \mathbb{P}\nabla\cdot(u^{(n)}\otimes u^{(n)})(s)\bigr)ds + \int_0^t G_{t-s} \mathbb{P}f(s)ds =\\ G_t u(0) - \int_0^t G_{t-s} \mathbb{P}\nabla\cdot(u\otimes u)(s)ds + \int_0^t G_{t-s} \mathbb{P}f(s)ds,
\end{multline*}
that is it is a solution.

% \begin{grayed}
% Let's denote for $\underline{a}\mathcal{X}=\supfun{\mathbb{R}}\supfun{f}\mathcal{X}$ for $a\in\mathbb{R}$, $f\in X\to\mathbb{R}$, $\lim_{\mathcal{X}}f=a$.

% Prove that it does not depend on the choice of~$f$:
% \end{grayed}

\begin{thm}
The heat semigroup $G_t$ is infinitely smoothing, that is $G_t f\in C^\infty$ for any $f:X\to L^1(\SUPER(\mathbb{R}))$ for $t>0$. (The norm is taken in~$\SUPER(\mathbb{R})$)
\end{thm}

\begin{proof}
There is only one zone for $G_t f$, because $G_t f(x)\ne 0$. So the classical result applies.

% \begin{grayed}
% Below the norms are taken in~$\SUPER(\mathbb{R})$.\fxnote{Need to prove that the used formulas are OK for this norm.}

% We will prove $\norm{e^{t\Delta}f}_{H^{s+r}}\le Ct^{-r/2}\norm{f}_{H^s}$, where the norms are taken in~$\SUPER(\mathbb{R})$.\fxnote{Details and how the thesis follows.}

% By uniform continuity of the argument\fxnote{Specify the argument.} of $G_t$ on closed inteval, its argument is $H^{s+r}$. So it is enough to show that $\norm{G_t}_{H^{s+r}\to H^s}\le Ct^{-r/2}$.

% Equilently\fxnote{Prove the equivalence.} for $k\ge 0$,
% \[ \norm{\nabla^k G_t f}_{L^2}\leq Ct^{-r/2}\norm{f}_{L^2}. \]

% Using the Fourier transform
% \[ \widehat{G_t f}(\xi) = \int_{\mathbb{R}^N}(1+\abs{\xi}^2)^{s+r}e^{-2\nu t\abs{\xi^2}}\abs{\hat{f}(\xi)}^2 d\xi. \]
% It's easy to see\fxnote{Show.} that $(1+\abs{\xi}^2)^{s+r}e^{-2\nu t\abs{\xi^2}}\le Ct^{-r}$. Thus the integral $\le Ct^{-r}\norm{f}_{H^s}^2$. Take square root to get the stated estimate.
% \end{grayed}

% It is known\fxnote{Reference.}, that $G_t$ is infinitely smoothing classical functions (and even distributions).

% So, $G_t f\in C^\infty$ for any classical~$f$.

% \fxnote{Formulate the proof exactly.}Let $F$ be a supersingular function. Let $\mathcal{X}$ be an ultrafilter or, equilently, a sequence of points converging to a point $x$.

% By the above shown interchange of limits,
% \[ \supfun{G_t}\underline{F}\mathcal{X} = G_t\lim_{x\to\mathcal{X}}f(x) = \lim_{x\to\mathcal{X}}G_t F(x). \]
% So $G_t$ is a limit of an infinitely smoothing function and so is infinitely smoothing.
\end{proof}

Therefore the integral~$u$ is $C^\infty$.

It is well-known that~$u$ is classical for some time interval starting from $t=0$. Therefore, there is a leftmost zone~$Z$ for integral arguments. If this is the only zone, we are done: the classical solution is easy (divide by $k_Z$). Otherwise, at the end time of the zone $u=0$, and we also know that the further in time solutions are classical.

\section{Further directions}

I could argue, that the requirement of $C^\infty$ for physical reasonableness is superfluous, because it is not used in the PDE derivation. So, the real physical problem remains unsolved: Is the solution unique if we require only existence of classical differential operators?

I also wonder why Clay Math considered physically unreasonable energy reaching infinity. The physically unreasonable assumption is not this but a force acting onto the fluid remaining the same no matter how the fluid moves. This may probably lead to infinite work, what is physically unreasonable.

So, whether the energy is uniformly bounded remains an open problem, without a clear physical hint.

Proving existence of classical $C^\infty$ solution moves the problem into classical analysis, where I don't claim to be an expert (despite my specialization in the university was ``math analysis'').

Also, I want to note that the real breakthrough is not this proof of existence of solutions, my theory of ordered semicategory actions (with generalized limit being its small part).

\section{The ChatGPT prompt}

The prompt is not displayed in~\cite{navier-stokes-chat} (apparently, due to a ChatGPT bug). So, the exact prompt has been lost. It seems that I asked ChatGPT to solve the Navier-Stokes Millennium Prize problem using my general topology result as an axiom.

\bibliographystyle{plain}
\bibliography{refs}

\end{document}