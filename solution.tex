\documentclass{amsart}
\usepackage{hyperref}
\usepackage[draft]{fixme}
\usepackage{xcolor}

\newenvironment{grayed}{\color{gray}}{\ignorespacesafterend}

\newtheorem{thm}{Theorem}
\newtheorem{lem}{Lemma}
\newtheorem{defn}{Definition}
\newtheorem{rem}{Remark}

\newcommand{\setcond}[2]{\left\{ #1 \mid #2 \right\}}
\DeclareMathOperator{\xlim}{xlim}
\DeclareMathOperator{\SUPER}{SUPER}

\begin{document}

\noindent
Ad: \href{https://science-dao.org}{Donate for science.}

\title{A Review of ChatGPT's Proof of Navier-Stokes Clay Math Millennium Prize Problem}

\author{Victor Porton, ORCID 0000-0001-7064-7975}

\email{\href{mailto:mailto:porton.victor@gmail.com}{porton.victor@gmail.com}}

\urladdr{\href{https://math.portonvictor.org}{https://math.portonvictor.org}}

\date{\today}

\begin{abstract}
Musing with ChatGPT for a full solution of the Navier-Stokes Millennium Prize problem, by usage of my general topology theorem proved before.
\end{abstract}

\maketitle

\section{Introduction}

(Because we may need $C^\infty$ to warrant integrability, it proves only existence of unique smooth solutions, but not uniqueness of all solutions. Oh, \href{https://grok.com/share/c2hhcmQtMw_8a8b4ca5-650d-4d7b-b1bc-e9dcfa6f4c7e}{here} it's said that existence of smooth solutions is enough.)

This is an attempt to elaborate and check the proof by ChatGPT~\cite{navier-stokes-chat} of the Navier-Stokes Clay Math Millennium Prize Problem, using the approach of generalized limits~\cite{limit} (more general generalized functions than distributions).

So, I did a half of work (the theorem that $\lim$ (limit) functional can be linearly extended to the space of all functions and that on the space of extended limits all operations can be extended with preservation of all algebraic identities). And ChatGPT did a half of work (applied known methods with addition of my this theorem).

Extaordinary claims need extraordinary evidence. The claim is a Millennium Prize Problem solution. The evidence of this ``functional extension'' theorem is an extraordinary evidence, because it is a new very powerful theorem in general topology.

I have a trouble to understand the ChatGPT's proof, because while being the world-best expert in general topology, I am no expert in differential equations. So, you may help.

\begin{grayed}
\section{Relation to distributions}

We define multiplication in the space of generalized limits, but well-behaved (fundamental theorem of calculus) integrals are in the space of distributions. (There are definite integrals in the space of generalized limits functions, but for them it seems that the fundamental theorem of calculus does not hold.) So, we need to immerse the space of distributions into the space of generalized limits, to freely operate with all three spaces: functions with real number values, distributions, and functions with generalized limit values.
\end{grayed}

\section{Problem statement}

It seems that ChatGPT  got \href{https://www.claymath.org/wp-content/uploads/2022/06/navierstokes.pdf}{the problem statement} somehow wrong.

We are going to prove existece (in the sense of classical real-number solutions) and smoothness~($C^\infty$) of solutions to Navier-Stokes equations on $\mathbb{R}^3$.
Let $u:\mathbb{R}^3\times\mathbb{R}\to\mathbb{R}^3$ and pressure $p(x,t)$.
\begin{align*}
\partial_t u - \nu\Delta u + (u \cdot \nabla) u + \nabla p &= 0;\\
\nabla \cdot u &= 0;\\
u({\cdot},0) &= u_0
\end{align*}
for viscosity $\nu>0$ and initial condition $u_0\in C^\infty$.

The solution constitutes an application for the Clay Math Millennium Problem Prize~\cite{navier-stokes-clay}.

\section{Integrals for supersingularities}

Let define definite integrals in the space of generalized limits functions as the difference of values of an antiderivative at the endpoints of the interval.

This definition suits only for integrals on a (closed) interval.

Each continuous function has an antiderivative and \href{https://chatgpt.com/s/t_69364f2479308191b79639101dbbfa48}{here} it's said that the terms under the integrals are continuous. The classic solution of the differential equations however, also require~$u$ to be continuous. So, this is not restrictive.

\section{Continuous functions on $\SUPER$}

The following theorem allows to extend Lebesgue integrals to the space of generalized limits functions in an easy way:

\begin{thm}
If $f:\mathbb{R}^n\to\SUPER(\mathbb{R})$ is continuous, then $\mathbb{R}^n\setminus\setcond{x\in\mathbb{R}^n}{f(x)\ne 0}$ splits into Lebesgue measurable sets (``zones''), each zone $Z=\setcond{ka}{k\in\mathbb{R}}$ for some $a\in\SUPER(\mathbb{R})$, where for boundary points~$x$ of the zones it holds that $f(x)=0$ for all $x$ on the boundary, then $f=kg$ for some $k\in\SUPER(\mathbb{R})$ and $g:\mathbb{R}^n\to\mathbb{R}$.
\end{thm}

\begin{proof}
Really, the distance between $c(a)$ and $c(b)$ for $\forall p,q\in\mathbb{R}:pa\ne qb$ is
\[ \inf_{p,q\in\mathbb{R}}|p a-q b| > 0, \]
if $p,q\ne 0$. Therefore boundaries of zones have $p,q=0$.

Subtracting boundaries from zones, we have zones (without loss of generality) being open sets and therefore Lebesgue measurable.
\end{proof}

\section{Lebesgue integrals}

Define Lebesgue integrals on the space of continuous generalized limits functions as \[ \int_A f(x) dx = \sum_{Z\in S}k_Z\int_Z a_Z(x) dx \] where $S$~is a set of zones.

It is easy to show that this definition does not depend on the choice of~$k_Z$ and~$a_Z$.

\begin{grayed}
Define Lebesgue integrals on the space of generalized limits functions \href{https://chatgpt.com/s/t_6937850a6b68819190a49e130f270ddb}{the same way} as Lebesgue integrals on the space of functions with real number values. This requires existence of (possibly infinite) supremum of a set of supersingularities.

Let $B$ be a set of supersingularities.\fxwarning{Elaborate the notation, prove details.}
\[ \sup B = \sup\setcond{\lim_{x\to 0} f(x)}{\lim_{x\to 0} f(x) \in B} \]

\begin{rem}
Lebesgue integral is infinite, if $\sup B$ wents infinite on at least one ultrafilter.\fxnote{Explain.}
\end{rem}
\end{grayed}

\begin{thm}
If an integral is absolutely convergent on a measurable set, then it is convergent.
\end{thm}

\begin{proof}
Obvious.

\begin{grayed}
\href{https://chatgpt.com/s/t_6937889e364881919fce0669e53a3115}{The same proof} as for real numbers.
\end{grayed}
\end{proof}

\section{Properties of Lebesgue integrals}

\begin{thm}
The fundamental theorem of calculus holds for integrals of continuous functions on the space of generalized limits functions:

Let $a<b$ be real numbers. Then:
\begin{enumerate}
\item If $f\in L^1([a,b])$ (absolutely integrable by Lebesgue) and
\[ F(x) = \int_a^x f(t) dt \quad\text{for } x\in [a,b]\]
then $F$ is absolutely continuous on $[a,b]$, differentiable almost everywhere on $[a,b]$, and $F'(x) = f(x)$ almost everywhere on $[a,b]$.
\item $G:[a,b]\to\SUPER(\mathbb{R})$ is absolutely continuous on $[a,b]$, $G'$ exists almost everywhere, and $G'(x) \in L^1([a,b])$ and
\[ G(x)-G(a) = \int_a^x G'(t) dt \quad\text{for } x\in [a,b].\]
\end{enumerate}
\end{thm}

\begin{proof}
Obvious.
\end{proof}

\section{Inhomogenious linear differential equations}

Theorem elaborated from \href{https://amirkdv.ca/files/inhom-pde.pdf}{here}.

\begin{defn}
\emph{Inhomogeneous linear evolution} with~$\Delta$ as the differential operator is
\begin{equation*}
\left\{
\begin{aligned}
(\partial_t - \Delta)u &= f;\\
u({\cdot},0) &= 0.
\end{aligned}
\right.
\end{equation*}
\end{defn}

\begin{thm}[First order Duhamel's principle]
Define auxiliary equations
\begin{equation*}
\left\{
\begin{aligned}
(\partial_t - \Delta)u_s &= 0;\\
u_s(x,s) &= f(x,s).
\end{aligned}
\right.
\end{equation*}
If the above family of equations have continuous in~$s$ solutions~$u_s(x,t)$ on supersingularities then the function
\[ u(x,t) = \int_0^t u_s(x,t) ds \]
is a continuous solution of the inhomogeneous linear evolution equation in supersingularities.
\end{thm}

\begin{proof}
Continuity of $u_s$ makes the integral defined.

$u({\cdot},0) = 0$ is obvious.

By the Leibniz rule for an integral with moving upper limit,
\[ \partial_t u(x,t) = \partial_t \int_0^t u_s(x,t) ds = u_t(x,t) + \int_0^t \partial_t u_s(x,t) ds. \]

$(\partial_t - \Delta)u(x,t) = \partial_t \int_0^t u_s(x,t) ds - \Delta \int_0^t u_s(x,t) ds =
u_t(x,t) + \int_0^t (\partial_t u_s(x,t) - \Delta u_s(x,t)) ds = f(x,t)$, so $(\partial_t - \Delta)u = f$.
\end{proof}

\section{Rewriting in integral form}

% The below proof assures uniqueness only of continuous solutions (because the integral requires a continuous argument). I didn't prove uniqueness of solutions. However, if at some point of time~$t_{\operatorname{crit}}$, the solutions ``split'' into both continuous and non-continuous~$u$, then it contradicts to a known theorem\fxnote{Cite.}, stating that under initial condition at~$t_{\operatorname{crit}}$ solutions are preserved to be smooth\fxwarning{Seems to be a wrong logic, because at~$t_{\operatorname{crit}}$ this solution can take only the initial value.}, so by contradiction we have full Clay Math problem solution.

% If at some point of time~$t_{\operatorname{crit}}$ speed~$u$ becomes non-continuous,

% Let's introduce a Banach norm on~$\mathbb{R}^3$:\fxwarning{Unused.}
% \[ \lVert y\rVert_2 = \sqrt{\sum_{i=1}^3 y_i^2}. \]
% \[ \lVert y\rVert = \sqrt{\sup\setcond{\limsup_{x\to 0} f(x)^2}{f:\mathbb{R}^3\to\mathbb{R}, \lim_{x\to 0} f(x) = y}}. \]

Let denote Leray projection as~$\mathbb{P}$.

\begin{lem}
If $u$ is a $C^\infty$ super-sin\-gu\-lar function, then \[ e^{\nu (t-s) \Delta} \mathbb{P} \nabla \cdot (u \otimes u)(s) \] is also a continuous super-singular function for $s\in[0,t]$.
\end{lem}

\begin{proof}
Obvious?
\end{proof}

\begin{defn}
I call the following defined for continuous functions taking supersingularities as values:
\begin{equation}
u(t) = e^{\nu t \Delta} u_0 - \int_0^t e^{\nu (t-s) \Delta} \mathbb{P} \nabla \cdot (u \otimes u)(s) ds.
\end{equation}
\emph{Supersingular integral solutions}\footnote{This formula is usually called \emph{mild (Duhamel) formulation}.} of Navier-Stokes equations.
\end{defn}

\begin{thm}
$C^\infty$ supersingular integral solutions of Navier-Stokes equations are exactly $C^\infty$ solutions of Navier-Stokes equations taken on the space of supersingularities.
\end{thm}

\begin{proof}
It is known that \[ \widehat{\mathbb{P}f}(\xi) = \left(I - \frac{\xi \otimes \xi}{|\xi|^2}\right)\hat{f}(\xi). \]

$\mathbb{P}$ is obviously a linear operator.

$\mathbb{P}\nabla q=0$ for any scalar~$q$.

$\mathbb{P}$ commutes with $\Delta$ and with the heat semigroup: 
$\mathbb{P}\Delta=\Delta\mathbb{P}$,
$\mathbb{P}e^{\nu t \Delta}=e^{\nu t \Delta}\mathbb{P}$. (It's clear from these eqaulities in the classical case and linear combinations for the zones.)

Apply $\mathbb{P}$ to the PDE. Since $\mathbb{P}\nabla p=0$, we get
\[ \mathbb{P}\partial_t u - \nu\mathbb{P}\Delta u - \mathbb{P}((u\cdot\nabla) u) = 0. \]

Because $u$ is divergence-free and $\mathbb{P}$ equals identity on divergence-free fields, $\mathbb{P}u=u$. $\mathbb{P}$ commutes with $\partial_t$ and $\Delta$.\fxnote{Need a symbolic proof.}

Thus we have the consequence
\[ \partial_t u - \nu\Delta u + \mathbb{P}((u\cdot\nabla) u) = 0. \]

We can write the nonlinear term as a divergence of a tensor:
\[ \mathbb{P}((u\cdot\nabla) u) = \nabla \cdot (\mathbb{P}(u\otimes u)), \]
because $(u\otimes u)_{ij}=u_i u_j$ and $\nabla \cdot (\mathbb{P}(u\otimes u)) = \sum_j\partial_j (u_i u_j)$.

Thus the PDE becomes: \[ \partial_t u - \nu\Delta u + \mathbb{P}\nabla \cdot (u\otimes u) = 0. \]

Rewrite this PDE as a system:
\[ \partial_t u - \nu\Delta u = -F(t),\quad F(t) = \mathbb{P}\nabla \cdot (u\otimes u). \]

For the linear homogeneous heat equation $\partial_t v - \nu\Delta v = 0$ with initial data $v(0)=v_0$ the solution is $v(t) = G_t v_0$.\fxnote{Need to define $G_t$.}

For the inhomogeneous equation, the Duhamel principle (variation-of-constants formula) gives the unique solution as
\[ u(t) = G_t u_0 + \int_0^t G_{t-s} \bigl( -F(s) \bigr) \, ds = G_t u_0 - \int_0^t G_{t-s} F(s) \, ds. \]

Substituting the value of~$F$ gives exactly the mild formula.

Now prove in the converse: if continuous~$u$ satisfies the mild equation, then it satisfies the PDE and we can recover a pressure~$p$.

Assume $u$ to be a continuous supersingular function.

Differentiate the mild equation in~$t$. Use facts:
\begin{itemize}
\item $\partial_t(e^{\nu t \Delta}u_0) = \nu \Delta e^{\nu t \Delta} u_0$;
\item $\frac{d}{dt}\int_0^t e^{\nu (t-s) \Delta} F(s) ds = F(t) + \int_0^t \nu e^{\nu (t-s) \Delta} F(s) ds$.
\end{itemize}
Apply this with $F(s) = \mathbb{P}\nabla \cdot (u\otimes u)(s)$. Differentiating mild gives:
\[
\partial_t u(t) = \nu \Delta e^{\nu t\Delta}u_0 - \mathbb{P}\nabla \cdot (u\otimes u)(t) + \int_0^t \nu\Delta e^{\nu (t-s)\Delta} \mathbb{P}\nabla \cdot (u\otimes u)(s) ds.
\]
Group terms:
\[ \partial_t u(t) - \nu \Delta u(t) = \mathbb{P}\nabla \cdot (u\otimes u)(t). \]
This is exactly the projected PDE.

Since $\mathbb{P}$ annihilates gradients, applying $(I-\mathbb{P})$ to the PDE recovers the pressure gradient term: start from the original PDE form:
\[ \partial_t u - \nu\Delta u + (u \cdot \nabla) u + \nabla p = 0 \]

We already have projection $\mathbb{P}$ of this equation satisfied. To obtain $\nabla p$ explicitly, take divergence of the PDE:
\[ \nabla \cdot \bigl(((u\cdot\nabla)u) + \nabla p\bigr) = 0 \]
because
\[ \nabla \cdot (\partial_t u - \nu\Delta u) = \partial_t(\nabla\cdot u) -\nabla\Delta(\nabla\cdot u) = 0 \]
when $\nabla\cdot u=0$ initially and evolution preserves it (or check separately).\fxnote{More detailed proof.}

Use $\nabla \cdot \bigl(u\cdot\nabla)u\bigr) = \sum_{i,j}\partial_i\partial_j(u_i u_j)$. Thus
\[ -\Delta p = \nabla \cdot \nabla (u\otimes u), \]
i.e.
\[ -\Delta p = \sum_{i,j}\partial_i\partial_j(u_i u_j)\quad\text{(Poisson)}. \]
??
\end{proof}

\section{TODO}

\fxwarning{The article is not finished.}

\bibliographystyle{plain}
\bibliography{refs}

\end{document}