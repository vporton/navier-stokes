\documentclass{amsart}
\usepackage{hyperref}
\usepackage[draft]{fixme}
\usepackage{xcolor}

\newenvironment{grayed}{\color{gray}}{\ignorespacesafterend}

\newcommand{\setcond}[2]{\left\{ #1 \mid #2 \right\}}
\DeclareMathOperator{\xlim}{xlim}

\begin{document}

\noindent
Ad: \href{https://science-dao.org}{Donate for science.}

\title{A Review of ChatGPT's Proof of Navier-Stokes Clay Math Millennium Prize Problem}

\author{Victor Porton, ORCID 0000-0001-7064-7975}

\email{\href{mailto:mailto:porton.victor@gmail.com}{porton.victor@gmail.com}}

\urladdr{\href{https://math.portonvictor.org}{https://math.portonvictor.org}}

\date{\today}

\begin{abstract}
Musing with ChatGPT for a full solution of the Navier-Stokes Millennium Prize problem, by usage of my general topology theorem proved before.
\end{abstract}

\maketitle

\section{Introduction}

This is an attempt to elaborate and check the proof by ChatGPT~\cite{navier-stokes-chat} of the Navier-Stokes Clay Math Millennium Prize Problem, using the approach of generalized limits~\cite{limit} (more general generalized functions than distributions).

So, I did a half of work (the theorem that $\lim$ (limit) functional can be linearly extended to the space of all functions and that on the space of extended limits all operations can be extended with preservation of all algebraic identities). And ChatGPT did a half of work (applied known methods with addition of my this theorem).

I have a trouble to understand the ChatGPT's proof, because while being the world-best expert in general topology, I am no expert in differential equations. So, you may help.

\begin{grayed}
\section{Relation to distributions}

We define multiplication in the space of generalized limits, but well-behaved (fundamental theorem of calculus) integrals are in the space of distributions. (There are definite integrals in the space of generalized limits functions, but for them it seems that the fundamental theorem of calculus does not hold.) So, we need to immerse the space of distributions into the space of generalized limits, to freely operate with all three spaces: functions with real number values, distributions, and functions with generalized limit values.
\end{grayed}

\section{Problem statement}

We are going to prove existece (in the sense of classical real-number solutions) and smoothness~($C^\infty$) of solutions to Navier-Stokes equations on $\mathbb{R}^3$.
Let $u:\mathbb{R}^3\times\mathbb{R}\to\mathbb{R}^3$ and pressure $p(x,t)$.
\begin{align*}
\partial_t u - \nu\Delta u + (u \cdot \nabla) u + \nabla p &= 0;\\
\nabla \cdot u &= 0;\\
u({\cdot},0) &= u_0
\end{align*}
for viscosity $\nu>0$ and initial condition $u_0\in C^\infty$.

The solution constitutes an application for the Clay Math Millennium Problem Prize~\cite{navier-stokes-clay}.

\section{Rewriting in integral form}

Let define definite integrals in the space of generalized limits functions as the difference of values of an antiderivative at the endpoints of the interval.

Each continuous function has an antiderivative and \href{https://chatgpt.com/s/t_69364f2479308191b79639101dbbfa48}{here} it's said that the terms under the integrals are continuous.

The below proof assures uniqueness only of continuous solutions (because the integral requires a continuous argument). I didn't prove uniqueness of solutions. However, if at some point of time~$t_{\operatorname{crit}}$, the solutions ``split'' into both continuous and non-continuous~$u$, then it contradicts to a known theorem\fxnote{Cite.}, stating that under initial condition at~$t_{\operatorname{crit}}$ solutions are preserved to be smooth\fxwarning{Seems to be a wrong logic, because at~$t_{\operatorname{crit}}$ this solution can take only the initial value.}, so by contradiction we have full Clay Math problem solution.

Let's introduce a Banach norm on~$\mathbb{R}^3$:
\[ \lVert y\rVert_2 = \sqrt{\sum_{i=1}^3 y_i^2}. \]
% \[ \lVert y\rVert = \sqrt{\sup\setcond{\limsup_{x\to 0} f(x)^2}{f:\mathbb{R}^3\to\mathbb{R}, \lim_{x\to 0} f(x) = y}}. \]

\section{TODO}

\fxwarning{The article is not finished.}

\bibliographystyle{plain}
\bibliography{refs}

\end{document}