\documentclass{amsart}
\usepackage{hyperref}
\usepackage[draft]{fixme}
\usepackage{xcolor}
\usepackage{mathrsfs}

\newenvironment{grayed}{\color{gray}}{\ignorespacesafterend}

\newtheorem{thm}{Theorem}
\newtheorem{lem}{Lemma}
\newtheorem{prop}{Proposition}
\newtheorem{defn}{Definition}
\newtheorem{rem}{Remark}

\newcommand{\setcond}[2]{\left\{ #1 \mid #2 \right\}}
\DeclareMathOperator{\xlim}{xlim}
\DeclareMathOperator{\SUPER}{SUPER}
\DeclareMathOperator{\SNG}{SNG}
\DeclareMathOperator{\fSNG}{fSNG}
\newcommand{\eqdef}{\overset{\mathrm{def}}{=}}
\newcommand{\subsets}{\mathscr{P}}
\newcommand{\norm}[1]{\left\lVert #1\right\rVert}
\newcommand{\abs}[1]{\left\lvert #1\right\rvert}
\newcommand{\supfun}[1]{\left\langle #1\right\rangle}

\hypersetup{pdftitle={A Proof of Existence of Smooth Classical Solutions of Na\-vi\-er-Sto\-kes Equations},
 pdfauthor={Victor Porton},
 pdfsubject={Na\-vi\-er-Sto\-kes equations},
 pdfkeywords={Na\-vi\-er-Sto\-kes equations existence and smoothness, Na\-vi\-er-Sto\-kes Clay Math Millennium Prize Problem, PDE, partial differential equations}}

\begin{document}

\noindent
Ad: \href{https://science-dao.org}{Donate for science.}

The most recent version of this document is available \href{https://github.com/vporton/navier-stokes}{here}.

\title[Na\-vi\-er-Sto\-kes]{A Proof of Existence of Smooth Classical Solutions of Na\-vi\-er-Sto\-kes Equations}

\author{Victor Porton, ORCID 0000-0001-7064-7975}

\email{\href{mailto:porton.victor@gmail.com}{porton.victor@gmail.com}}

\urladdr{\href{https://math.portonvictor.org}{https://math.portonvictor.org}}

\date{\today}

\begin{abstract}
A proof of existence of classical smooth solutions of the Na\-vi\-er-Sto\-kes Millennium Prize problem, by usage of my general topology theorem proved before. This constitutes a solution of the Millennium Prize problem.
\end{abstract}

\subjclass[2020]{35Q30, 35A01, 35A09, 35A25, 35D99}

\maketitle

\section{Introduction}

This is a rectified proof by ChatGPT~\cite{navier-stokes-chat} of the Na\-vi\-er-Sto\-kes Clay Math Millennium Prize Problem~\cite{navier-stokes-clay}, using the approach of generalized limits~\cite{limit} (more general generalized functions than distributions) suggested to the LLM by me. In fact, that ChatGPT's proof didn't settle the energy inequality, but only existence of solutions, despite it claimed~\cite{navier-stokes-chat} a full solution.

In my approach multiplication of every functions is defined, what allows to write formulas undefined for distributions.

So, I did a half of the work (the theorem that $\lim$ (limit) functional can be linearly extended to the space of all functions and that on the space of extended limits all continuous operations can be extended with preservation of all algebraic identities). And ChatGPT did a half of work (applied known methods with addition of my this theorem).

Extraordinary claims need extraordinary evidence. The claim is a Millennium Prize Problem solution. The evidence of this ``functional extension'' theorem is an extraordinary evidence, because it is a new very powerful theorem in general topology and its proof is extraordinarily longer than a noob would expect.

I had a little trouble to understand the ChatGPT's proof, because while being the world-best expert in general topology, I am no expert in differential equations. But I was able to write down my more detailed (and in parts more elegant than ChatGPT's) proof.

\section{Problem statement}

We are going to prove existence (in the sense of classical real-number solutions) and smoothness~($C^\infty$) of solutions to Na\-vi\-er-Sto\-kes equations on $\mathbb{R}^N$ ($N\in\mathbb{N}$).
Let $u(x,t):\mathbb{R}^N\times[0;\infty)\to\mathbb{R}^N$ and pressure $p(x,t):\mathbb{R}^N\times[0;\infty)\to\mathbb{R}$.
\begin{align*}
\partial_t u - \nu\Delta u + (u \cdot \nabla) u + \nabla p &= f ;\\
\nabla \cdot u &= 0;\\
u({\cdot},0) &= u_0
\end{align*}
for viscosity $\nu>0$, $f(t,x)\in C^\infty$ and initial condition $u_0(x)\in C^\infty$.

\section{The limit extension theorems}

We will consider functions $\mathbb{R}^P\to\mathbb{R}^Q$ or sequences $\mathbb{N}\to\mathbb{R}^Q$ for some $P,Q\in\mathbb{N}$. We define \emph{generalized limits} of such functions at points of $\mathbb{R}^P$ (and of sequences at infinity) by the formula~(1) in~\cite[p.~6]{limit}:
\begin{itemize}
\item For the group~$G$ in~\cite{limit} take the group of translations of $\mathbb{R}^P$ (or, alternatively, its isometries). For limits of sequences at infinity take~$G$ to be the identity permutations group.
\item On~$\mathbb{R}^Q$ we will consider the proximity relation~$\mathrm{X}$:
\[ A\mathrel{\mathrm{X}}B \eqdef \inf_{x\in A,y\in B}\abs{x-y}=0,\quad A,B\in\mathbb{R}^Q. \]

It's easy to show that for this relation, if considered as a funcoid~\cite{volume-1-edition1}, $\mathrm{X}^{-1}\circ\mathrm{X}=\mathrm{X}$ and consequently formula~(3) in~\cite{limit} by~\cite[proposition~6, p.~6]{limit} is valid.
\item It's obvious that $\mathrm{X}$ (as a funcoid~\cite[chapter~7, p.~144]{volume-1-edition1}) commutes with elements of~$G$.
\item Thus generalized limit is defined properly.
\end{itemize}

The generalized limit is defined for every function at all points (or for sequences at positive infinity).

In~\cite{limit} the set of all generalized limits over $X$ we denote as $\SNG(X)$. We can assume $X\subseteq\SNG(X)$ (by injectivity of $\tau$~\cite[p.~7]{limit}, using that $\mathbb{R}^Q$ is Hausdorff). When the usual limit exists, it is equal to the generalized limit~\cite[corollary 3, p.~7]{limit}.

We have all derivatives on the set~$X$ to be always defined taking values in~$\SNG(X)$.

We have all continuous finitary operations on~$\mathbb{R}^Q$ extendable to~$\fSNG({-},\mathbb{R}^Q)$ \cite[section 9, p.~8]{limit}. For all such operations, algebraic identities ($f(\dots)=g(\dots)$) on~$\fSNG({-},X)$ are the same as on~$X$~\cite[theorem 6, p.~10]{limit}, particularly generalized limit is a linear extension of the usual limit.

As shown in \cite[section~12, p.~11]{limit}, on $\fSNG({-},X)$ there is defined funcoid structure, what allows to take limits on it.

\section{Continuous supersingular functions with values in}

Let $X$ be some Lebesgue measurable set and some element $0\in X$ (not necessarily real zero or zero vector).

\begin{defn}
Let $a,b\in\SNG(X)$. Then for limits of functions $Y\to X$:
\[ a\le b \eqdef \exists f,g:Y\to X:\left(\lim_{x\to 0}f(x)=a\land\lim_{x\to 0}g(x)=b\land f\le g\right). \]
\end{defn}

The above easily can be easily extended to $a,b\in\SUPER(X)$.

\begin{prop}
The above defined is a partial order on~$\SUPER(\mathbb{R})$.
\end{prop}

\begin{proof}
Let
\[ a,b\in\underbrace{\SNG(\dots\SNG(\mathbb{R})\dots)}_{\text{Apply $\SNG$ $K$ times}}. \]

By induction, we have:

Transitivity and reflexivity are obvious.

For antisymmetry, let $a\leq b$ and $b\leq a$. We have $\lim_{x\to 0}|g(x)-f(x)|\geq 0$ and $\lim_{x\to 0}|g(x)-f(x)|\leq 0$. So, $\lim_{x\to 0}|g(x)-f(x)|=0$ and therefore $a=b$.
\end{proof}

\begin{thm}
For each $S\in\subsets\underbrace{\SNG(\dots\SNG(\mathbb{R})\dots)}_{\text{Finite number of times}}$, such that $\forall a\in S:a\ge B$ for some constant $B\in\mathbb{R}$, there exists $\inf S\in\SUPER(\mathbb{R})$.
\end{thm}

\begin{proof}
For the first level of the limit (of $\mathbb{R}$-va\-lu\-ed functions) we use this reasoning:

Choose
\[ F\in\prod_{a\in S}\setcond{f:X\to\mathbb{R}}{\lim_{x\to 0}f(x)=a}. \]
Let
\[ m=\lim_{x\to 0}\left(x\mapsto\inf_{f\in F}\max\{f(x),B\}\right). \]
($\inf_{f\in F}\max\{f(x),B\}$ exists because this set is bounded from below.) We will show that $m$ is the infimum of~$S$.

It's obvious that $\forall a\in S:m\le a$ and $\forall a\in S:x\le a\Rightarrow x\le m$.

For $\SNG(\dots\SNG(\mathbb{R})\dots)$ (finite number of limits) the same reasoning applies inductively.
\end{proof}

The following theorem allows to extend Lebesgue integrals to the space of generalized limits functions in an easy way:

\begin{thm}
If a set $X$ is Lebesgue measurable and $f:fSNG({-},\mathbb{R}^N)$ is continuous, then:
\begin{enumerate}
\item $\setcond{x\in X}{f(x)\ne 0}$ is partitioned into Lebesgue measurable sets (\emph{zones})
  \[ \setcond{Z_k}{k\in\SUPER(\mathbb{R}^N)\setminus\{0\}}, \]
  where zone
  \[ Z_k=\setcond{x\in X}{\exists a\in\mathbb{R}\setminus\{0\}:f(x)=ka}. \]
\item Each $Z_k$ is an open set in~$X$.
\item $f(x)=0$ for each $x$ on the boundary $X\cap\partial Z$ of each zone.
\item\label{as-func-mult} $f(x)=k_Z g_Z(x)$ for each $x\in X\cap\overline{Z}$ for some $k_Z\in\SUPER(\mathbb{R}^N)\setminus\{0\}$ and $g_Z:X\cap\overline{Z}\to\mathbb{R}$ for each zone $Z$.
\end{enumerate}
\end{thm}

\begin{proof}
To prove that $f(a)=0$ on boundaries of zones, consider two only possible cases for $a\in X\cap\partial Z_k$:
\begin{enumerate}
\item ``$a\in\overline{Z_k}$ and $\forall\epsilon>0\exists x\in\mathbb{R}:(\abs{a-x}<\epsilon\land f(x)=0)$.'' This is possible only if $f(a)=0$ due to continuity of~$f$.

\item ``$a\in\overline{Z_k}$ and $\forall\epsilon>0\exists x\in\mathbb{R}:(\abs{a-x}<\epsilon\land f(x)=k'g(x))$, where $k'\in\SUPER(\mathbb{R}^N)\setminus\{0\}$ and $g:X\cap\overline{Z'}\to\mathbb{R}$ where zone $Z'\ne Z_k$.''
$f(x)$ belongs to the same zone $Z_k$, unless $f(a)=0$, because otherwise we would have $kf(x)=k'f(x)$ for $k$ and~$k'$ from different zones $Z_k$ and~$Z'$, respectively.
\end{enumerate}

$Z_k$ doesn't intersect $\partial Z_k$ and is, therefore, open. Obviously zones don't intersect and their union is $\setcond{x\in X}{f(x)\ne 0}$.

Item~\ref{as-func-mult} is obvious.

We have zones (without loss of generality) being open sets (because not containing points from their boundaries) and therefore Lebesgue measurable.
\end{proof}

\begin{defn}
I call the zone with $k\in\mathbb{R}^N\setminus\{0\}$ \emph{the classical zone}.
\end{defn}

\section{Lebesgue integrals}

By a \emph{supersingular function} or a solution in \emph{supersingularities} I mean a function $f:\fSNG(X,\mathbb{R})$ (where~$X$ is a suitable space), see~\cite{limit}.

\begin{rem}
(Lebesgue) integrals could be instead defined on the space of generalized limits functions the same way as Lebesgue integrals on the space of functions with real number values. This requires existence of (possibly infinite) supremum of a set of supersingularities. But for our purposes it is enough to define integrals on the space of continuous generalized limits functions, and this special case is much simpler to prove properties of.
\end{rem}

Define Lebesgue integrals on the space of continuous generalized limits functions as \[ \int_A f(x) dx = \sum_{Z\in S}k_Z\int_Z a_Z(x) dx \] if each Lebesgue integral on the right side exists and the sum is absolutely convergent (and so does not depend on the order of summation), where $S$~is a set of zones.

It is easy to show that this definition does not depend on the choice of~$k_Z$ and~$a_Z$.

Below, we will, contrary to the usual notation, denote~$L^1(X)$ the space of functions on~$X$ integrable by the above formula, not all Lebesgue integrable functions.

\begin{thm}
If an integral is absolutely convergent on a measurable set, then it is convergent.
\end{thm}

\begin{proof}
Obvious consequence of the same property for Lebesgue integrals on the space of functions with real number values.
\end{proof}

\subsection{Properties of Lebesgue integrals}

\begin{thm}
The fundamental theorem of calculus holds for integrals of continuous functions on the space of generalized limits functions:

Let $a<b$ be real numbers. Then:
\begin{enumerate}
\item If $f\in L^1([a,b])$ (absolutely integrable by Lebesgue) and
\[ F(x) = \int_a^x f(t) dt \quad\text{for } x\in [a,b]\]
then $F$ is absolutely continuous on $[a,b]$, differentiable almost everywhere on $[a,b]$, and $F'(x) = f(x)$ almost everywhere on $[a,b]$.
\item $G\in fSNG([a,b],\mathbb{R})$ is absolutely continuous on $[a,b]$, $G'$ exists almost everywhere, and $G'(x) \in L^1([a,b])$ and
\[ G(x)-G(a) = \int_a^x G'(t) dt \quad\text{for } x\in [a,b].\]
\end{enumerate}
\end{thm}

\begin{proof}
Obvious consequence of the same property for Lebesgue integrals on the space of functions with real number values (we use an absolutely convergent series).
\end{proof}

\section{Inhomogenious linear differential equations}

Theorem elaborated from~\cite{duhamel-chat}.

\begin{defn}
\emph{Inhomogeneous linear evolution} with~$\Delta$ as the differential operator is
\begin{equation*}
\left\{
\begin{aligned}
(\partial_t - \Delta)u &= f;\\
u({\cdot},0) &= 0.
\end{aligned}
\right.
\end{equation*}
\end{defn}

\begin{thm}[First order Duhamel's principle]
Define auxiliary equations
\begin{equation*}
\left\{
\begin{aligned}
(\partial_t - \Delta)u_s &= 0;\\
u_s(x,s) &= f(x,s).
\end{aligned}
\right.
\end{equation*}
If the above family of equations have continuous in~$s$ solutions~$u_s(x,t)$ in supersingularities then the function
\[ u(x,t) = \int_0^t u_s(x,t) ds \]
is a continuous solution of the inhomogeneous linear evolution equation in supersingularities.
\end{thm}

\begin{proof}
Continuity of $u_s$ makes the integral defined.

$u({\cdot},0) = 0$ is obvious.

By the Leibniz rule for an integral with moving upper limit,
\[ \partial_t u(x,t) = \partial_t \int_0^t u_s(x,t) ds = u_t(x,t) + \int_0^t \partial_t u_s(x,t) ds. \]
\begin{multline*}
(\partial_t - \Delta)u(x,t) = \partial_t \int_0^t u_s(x,t) ds - \Delta \int_0^t u_s(x,t) ds =\\
u_t(x,t) + \int_0^t (\partial_t u_s(x,t) - \Delta u_s(x,t)) ds = f(x,t),
\end{multline*}
so $(\partial_t - \Delta)u = f$.
\end{proof}

\section{Rewriting in integral form}




Let denote Leray projection as~$\mathbb{P}$.

\begin{lem}
If $u$ is a $C^\infty$ super-sin\-gu\-lar function, then \[ e^{\nu (t-s) \Delta} \mathbb{P} \nabla \cdot (u \otimes u)(s) \] is also a continuous super-sin\-gu\-lar function for $s\in[0,t]$.
\end{lem}

\begin{proof}
Obvious.
\end{proof}

\begin{defn}
I call the following defined for supersingular continuous functions:
\begin{equation}
u(t) = e^{\nu t \Delta} u_0 - \int_0^t e^{\nu (t-s) \Delta} \mathbb{P} \nabla \cdot (u \otimes u)(s) ds
+ \int_0^t e^{\nu (t-s) \Delta} \mathbb{P} f(s) ds
\end{equation}
\emph{supersingular mild solutions}\footnote{This formula is usually called \emph{mild (Duhamel) formulation}.} of Na\-vi\-er-Sto\-kes equations.
\end{defn}


\begin{thm}
$C^\infty$ supersingular mild solutions of Na\-vi\-er-Sto\-kes equations are exactly $C^\infty$ solutions of Na\-vi\-er-Sto\-kes equations (without the requirement that~$p$ is $C^\infty$) in supersingularities.
\end{thm}

\begin{proof}
It is known that \[ \widehat{\mathbb{P}f}(\xi) = \left(I - \frac{\xi \otimes \xi}{|\xi|^2}\right)\hat{f}(\xi). \]

$\mathbb{P}$ is obviously a linear operator.

$\mathbb{P}\nabla q=0$ for any scalar~$q$.

$\mathbb{P}$ commutes with $\Delta$ and with the heat semigroup: 
$\mathbb{P}\Delta=\Delta\mathbb{P}$,
$\mathbb{P}e^{\nu t \Delta}=e^{\nu t \Delta}\mathbb{P}$. (It's clear from these equalities in the classical case and linear combinations for the zones.)

Apply $\mathbb{P}$ to the PDE. Since $\mathbb{P}\nabla p=0$, we get
\[ \mathbb{P}\partial_t u - \nu\mathbb{P}\Delta u - \mathbb{P}((u\cdot\nabla) u) = \mathbb{P}f. \]

Because $u$ is divergence-free and $\mathbb{P}$ equals identity on divergence-free fields, $\mathbb{P}u=u$. $\mathbb{P}$ commutes with $\partial_t$ and $\Delta$.

Thus we have the consequence
\[ \partial_t u - \nu\Delta u + \mathbb{P}((u\cdot\nabla) u) = \mathbb{P}f. \]

We can write the nonlinear term as a divergence of a tensor:
\[ (u\cdot\nabla) u = \nabla \cdot (u\otimes u), \]
because $(u\otimes u)_{ij}=u_i u_j$ and $\nabla \cdot (u\otimes u) = \sum_j\partial_j (u_i u_j)$.

Thus the PDE becomes: \[ \partial_t u - \nu\Delta u + \mathbb{P}\nabla \cdot (u\otimes u) = \mathbb{P}f. \]

Rewrite this PDE as a system:
\[ \partial_t u - \nu\Delta u = -F(t) + \mathbb{P}f(t),\quad F(t) = \mathbb{P}\nabla \cdot (u\otimes u). \]

Denote $G_t = e^{\nu t\Delta}$.

For the linear homogeneous heat equation $\partial_t v - \nu\Delta v = 0$ with initial data $v(0)=v_0$ the solution is $v(t) = G_t v_0$.

For the inhomogeneous equation, the Duhamel principle (variation-of-constants formula) gives the unique solution as
\[ u(t) = G_t u_0 + \int_0^t G_{t-s} \bigl( -F(s) + \mathbb{P}f(s) \bigr) \, ds = G_t u_0 - \int_0^t G_{t-s} F(s) \, ds + \int_0^t G_{t-s} \mathbb{P}f(s) \, ds. \]

Substituting the value of~$F$ gives exactly the mild formula.

Now prove in the converse: if continuous~$u$ satisfies the mild equation, then it satisfies the PDE and we can recover a pressure~$p$. The below proof is correct only almost everywhere in time. But since the functions in question are~$C^\infty$, it extends to everywhere.

Assume $u$ to be a continuous supersingular function.

Differentiate the mild equation in~$t$. Use facts:
\begin{itemize}
\item $\partial_t(e^{\nu t \Delta}u_0) = \nu \Delta e^{\nu t \Delta} u_0$;
\item $\frac{d}{dt}\int_0^t e^{\nu (t-s) \Delta} F(s) ds = F(t) + \int_0^t \nu e^{\nu (t-s) \Delta} F(s) ds$.
\end{itemize}
Apply this with $F(s) = \mathbb{P}\nabla \cdot (u\otimes u)(s)$. Differentiating mild gives:
\begin{multline*}
\partial_t u(t) = \nu \Delta e^{\nu t\Delta}u_0 - \mathbb{P}\nabla \cdot (u\otimes u)(t) + {} \\ \int_0^t \nu\Delta e^{\nu (t-s)\Delta} \mathbb{P}\nabla \cdot (u\otimes u)(s) ds + \mathbb{P}f(t) - \int_0^t \nu\Delta e^{\nu (t-s)\Delta} \mathbb{P}f(s) ds.
\end{multline*}
Group terms:
\[ \partial_t u(t) - \nu \Delta u(t) = \mathbb{P}\nabla \cdot (u\otimes u)(t) + \mathbb{P}f(t). \]
This is exactly the projected PDE.

Since $\mathbb{P}$ annihilates gradients, applying $(I-\mathbb{P})$ to the PDE recovers the pressure gradient term: start from the original PDE form:
\[ \partial_t u - \nu\Delta u + (u \cdot \nabla) u + \nabla p = f. \]

We already have projection $\mathbb{P}$ of this equation satisfied. To obtain $\nabla p$ explicitly, take divergence of the PDE:
\[ \nabla \cdot \bigl(((u\cdot\nabla)u) + \nabla p - f\bigr) = 0 \]
because
\[ \nabla \cdot (\partial_t u - \nu\Delta u) = \partial_t(\nabla\cdot u) -\nu\Delta(\nabla\cdot u) = 0 \]
when $\nabla\cdot u=0$ initially and evolution preserves it (or check separately).

Use $\nabla\cdot(u\cdot\nabla)u = \sum_{i,j}\partial_i\partial_j(u_i u_j)$. Thus
\[ -\Delta p = \nabla \cdot \nabla (u\otimes u) - \nabla \cdot f\quad\text{(Poisson)}. \]
It is common knowledge that the Poisson equation's solution is $C^\infty$ if $u$ and $f$ are $C^\infty$.

Because $u$ is smooth, we can split this equality into zones, and it is true in every zone.
On borders of zones, it can be constructed as a limit of the Poisson equation's solution in a zone. It does not depend on choice of the zone we take limit in, because~$u\in C^\infty$. Outside of zones it's zero.
So, we have supersingular solution, too.
\end{proof}

\section{The sequence of approximate solutions}

Define the Picard iteration mapping~$\mathcal{T}$ on ti\-me-de\-pen\-dent $C^\infty$ vector fields by
\begin{equation*}
(\mathcal{T}v)(t) = G_t u_0 - \int_0^t G_{t-s}\mathbb{P}\nabla\cdot(v\otimes v)(s)ds + \int_0^t G_{t-s}\mathbb{P}f(s)ds.
\end{equation*}
So, we have a sequence
\[ u^{(0)}(t) = u_0,\quad u^{(n+1)}(t) = (\mathcal{T}u^{(n)})(t). \]

Note that the integrand is~$L^1$ because $\nabla\cdot(v\otimes v)$ is~$C^\infty$.

Each element of this sequence is a continuous supersingular function.

Try as the solution $u(t)=\lim_{n\to\infty}u^{(n)}(t)$ the generalized limit of this sequence.

\section{Interchange of limit and integration}

Let $\lim$ be our extended limit. On each (closed) zone (a closed interval), $\lim$ is a bounded linear operator. Therefore (because it is a linear bounded operator~\cite{int-banach} $\lim$~can be interchanged with integrals.)
It also maps absolutely convergent series to absolutely convergent series, therefore this extends to the entire closed interval. Thus we have:

\begin{multline*}
u(t) = \lim_{n\to\infty}u^{(n)}(t) =\\ G_t u(0) - \int_0^t \lim_{n\to\infty}\bigl(G_{t-s} \mathbb{P}\nabla\cdot(u^{(n)}\otimes u^{(n)})(s)\bigr)ds + \int_0^t G_{t-s} \mathbb{P}f(s)ds =\\ G_t u(0) - \int_0^t G_{t-s} \mathbb{P}\nabla\cdot(u\otimes u)(s)ds + \int_0^t G_{t-s} \mathbb{P}f(s)ds,
\end{multline*}
that is it is a solution.

\begin{thm}
The heat semigroup $G_t$ is infinitely smoothing, that is $G_t f\in C^\infty$ for any $f:X\to L^1$ on $\fSNG({-},\mathbb{R})$ for $t>0$. (The norm is taken in~$\fSNG({-},\mathbb{R})$.)
\end{thm}

\begin{proof}
Apply the classical result for each closed zone~$\overline{Z}$ of~$G_t f$.
\end{proof}

Therefore the integral~$u$ is $C^\infty$ (on the set $\fSNG({-},\mathbb{R}^N)$).

It remains to prove that $u$ takes on values in~$\mathbb{R}^N$ (is classical).

Rewrite the equation as:
\begin{equation}\label{to-classical}
G_t u(0) - u(t) = \int_0^t G_{t-s} \bigl(\mathbb{P}\nabla\cdot(u\otimes u)(s) - \mathbb{P}f(s)\bigr)ds.
\end{equation}

It is well-known that~$u$ is classical for some time interval starting from $t=0$. Suppose that at the time $t_{\operatorname{crit}}$, the solution becomes non-clas\-si\-cal. This is possible only at a border of a ``classical'' zone (the integrand in~\eqref{to-classical} is equal to~$0$ at $t=t_{\operatorname{crit}}$), therefore the left derivative (and by $C^\infty$, right derivative, too) of the integral is in~$\mathbb{R}$ for $t\in[t_{\operatorname{crit}}-\epsilon,t_{\operatorname{crit}}]$. This warrants that the derivative on $[t_{\operatorname{crit}}-\epsilon,t_{\operatorname{crit}}]$ is limited and therefore the integral takes on $[t_{\operatorname{crit}}-\epsilon,t_{\operatorname{crit}}]$ values in~$\mathbb{R}$. Therefore the integral is in~$\mathbb{R}$ at~$t=t_{\operatorname{crit}}$.

\section{The energy inequality}

First, note that it is well-known that the classical solutions are unique.

To finish the solution of the Clay Math Millennium Prize Problem, we need to prove the energy inequality~\cite{navier-stokes-clay} in the 3D case.

But it is well-known that the energy inequality holds for classical solutions. For a reference, see~\cite{navier-stokes-energy}.

\section{Further directions}

I could argue, that the requirement of $C^\infty$ for physical reasonableness is superfluous, because it is not used in the PDE derivation. So, the real physical problem remains unsolved: Is the solution unique if we require only existence of classical differential operators?

Also, I want to note that the real breakthrough is not this proof of existence of solutions, but my theory of ordered semicategory actions (with generalized limit being its small part).

\section{The ChatGPT prompt}

The prompt is not displayed in~\cite{navier-stokes-chat} (apparently, due to a ChatGPT bug). So, the exact prompt has been lost. It seems that I asked ChatGPT to solve the Na\-vi\-er-Sto\-kes Millennium Prize problem positively using my general topology result as an axiom.

\bibliographystyle{plain}
\bibliography{refs}

\end{document}